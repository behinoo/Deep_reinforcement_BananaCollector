\documentclass[]{article}

\usepackage{graphicx}
\usepackage{subfig}
% Title Page
\title{Navigation Using Deep Q learning }
\author{}


\begin{document}
\maketitle


\section{Goal}


The goal of the agent is to navigate (collect yellow bananas while avoid blue bananas!) in a large, square world.\
A reward of +1 is provided for collecting a yellow banana, and a reward of -1 is provided for collecting a blue banana.  Thus, the goal of your agent is to collect as many yellow bananas as possible while avoiding blue bananas.  

The state space has 37 dimensions and contains the agent's velocity, along with ray-based perception of objects around agent's forward direction.  Given this information, the agent has to learn how to best select actions.  Four discrete actions are available, corresponding to:

\begin{list}{}{}
	\item  0 - move forward.
	\item  1 - move backward.
	\item  2 - move left.
	\item  3 - move right.
\end{list}
The task is episodic, and in order to solve the environment, your agent must get an average score of +13 over 100 consecutive episodes. In order to compare different techniques the training was stopped once the score reached an average of 13. 
 

\section{Method}
Two different approaches were used in this project.
Q-learning (DQN) and Double Q-learning (DDQN).


A deep Q network (DQN) is a multi-layered neural network that for a given state s outputs a vector of action values. I used the fixed target network and experience replay similar to \cite{mnih2015human}
The max operator in standard Q-learning and DQN, uses the same values both to select and to evaluate
an action. This makes it more likely to select overestimated values, resulting in overoptimistic value estimates. To prevent
this, we can decouple the selection from the evaluation. In Double Q-learning, two value functions
are learned by assigning experiences randomly to update
one of the two value functions, resulting in two sets of
weights, θ and θ'. For each update, one set of weights is used to determine the greedy policy and the other to determine its \cite{hasselt2010double}. 

Neural network with following architecture was used to approximate Q value. 
\begin{figure}
	\centering
	\includegraphics[width=0.7\linewidth]{Picture1}
	\caption{Neural network Architecture}
	\label{fig:picture1}
\end{figure}

\section{Results}
The agent based DQN implementation could achieve a score of 13 after 542 episodes and the agent based on DDQN could achieve a score of 13 after 509 episodes.

\begin{figure}
	\centering
	\subfloat[DQN]{\includegraphics[width=0.3\textwidth]{DQN}\label{fig:f1}}
	\hfill
	\subfloat[DDQN]{\includegraphics[width=0.3\textwidth]{DDQN}\label{fig:f2}}
	\hfill
	\subfloat[DQN vs DDQN]{\includegraphics[width=0.3\textwidth]{Compare}\label{fig:f2}}
	\caption{DDQN vs DQN  average scores  }
\end{figure}

DDQN shows slight improvement in compare to DQN.  Hyper parameter tuning can also improve the performance of the agent. 

\section{Improvement}
Other techniques could be explored to improve the current model such as Prioritized DDQN,Dueling DDQN, A3C, Distributional DQN and rainbow \cite{hessel2017rainbow}


\clearpage
\bibliographystyle{plain}

\bibliography{Report}
\end{document}          
